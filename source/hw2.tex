\documentclass[12pt]{article}
\usepackage[tagged, highstructure]{accessibility}
\usepackage[english]{babel}
\usepackage[utf8x]{inputenc}
\usepackage[T1]{fontenc}
\usepackage[margin=1in]{geometry}
\usepackage{scribe}
\usepackage{listings}
\usepackage{natbib,verbatim}
\usepackage{amsmath,amssymb,amsfonts}
\usepackage{hyperref}
\hypersetup{
    colorlinks=true,
    linkcolor=blue,
    filecolor=magenta,      
    urlcolor=magenta,
    pdftitle={Course Syllabus},
    pdfauthor={Nisha Chandramoorthy},
    pdflang={en-US}
}

%\Scribe{Your Name}
\title{Syllabus-37780}
\Lecturer{Nisha Chandramoorthy (nishac@uchicago.edu)}
\LectureNumber{\begin{large} Mathematics of Generative Models \end{large}}
\LectureDate{}
\LectureTitle{\begin{large}Homework 2\end{large}}

\lstset{style=mystyle}

\begin{document}
\MakeScribeTop

In this homework, we will work on flow matching models in 2D. 
Consider the ODE 
\begin{equation}
    \label{eq:flow}
    \dfrac{dx}{dt} = v(x,t) \quad t \in [0,1]. 
\end{equation}
The solution to the ODE above at time $t$ starting with the initial condition $x \in \mathbb{R}^2$ is given by 
$\varphi^t(x) \in \mathbb{R}^2.$ The function $(t,x) \to \varphi^t(x)$ is also called the flow of the vector field $v.$ The flow, $\varphi^t,$ is deterministic, but we will consider the initial conditions to be random variables taking values in $\mathbb{R}^2$. Let $X_0$ be the random variable denoting the initial condition and having a distribution, $\mu,$ with probability density $\rho_0.$ Let $X_t = \varphi^t(X_0)$, with a probability density, $\rho_t.$ We have already seen that $\rho_t$ solves the continuity equation given by,
\begin{equation}
	\label{eq:continuity}
	\dfrac{\partial \rho_t}{\partial t} = - \mathrm{div}(\rho_t \: v(\cdot,t)).
\end{equation}
At time 1, we want the random variable $X_1$ to have the desired distribution $p_\mathrm{data}$, which may not have a density. Flow matching models learn a time dependent vector field $v$ so that the density $\rho_1$ is an approximate density of $p_\mathrm{data}$. But, they must learn $v$ using only independent samples $\{x^{(1)}, \cdots, x^{(m)}\}$ from $p_\mathrm{data}$ and no other information about $p_\mathrm{data}.$ Let $\hat{p}_{\mathrm{data},m} = \mathrm{Unif}\{ x^{(1)}, \cdots, x^{(m)}\}$ be the empirical distribution over the given samples from the target distribution, $p_\mathrm{data}$. Similarly, let $\hat{\mu}_m$ denote the empirical distribution over $m$ independent samples from $\mu.$ For this homework, take $p_\mathrm{data} = \frac{1}{4}\mathcal{N}(\mu_1, \Sigma_1) + \frac{3}{4}\mathcal{N}(\mu_2, \Sigma_2)$ with means $\mu_1 = (-2, 0)^\top,$ $\mu_2 = (2, 0)^\top$ and covariances $\Sigma_1 = \begin{pmatrix} 2 & 0 \\ 0 & 0.5 \end{pmatrix},$ $\Sigma_2 = I_2.$


\begin{enumerate}
	\item Let $Z = (X_0, X_1)$ be a random variable distributed according to $\hat{\mu}_m \times \hat{p}_\mathrm{data}.$  
    Given $Z = (x_0, x_1),$ suppose we choose an interpolation path $\varphi^t(x_0|Z=(x_0,x_1)) = \alpha_t x_0 + \beta_t x_1.$ 
    What is the conditional vector field, $u(\cdot, t),$? (1 point)
    \item What is the solution of the continuity equation for $u(\cdot, t)$? (1 point)
    \item Let $v_\theta$ be a neural network approximation of $v$. The flow matching loss is 
    $$ l_\mathrm{FM}(\theta) = \int \|v(t,x) - v_\theta(t,x)\|^2\: \rho_t(x)\: dx\: dt,$$
    where the integrals are approximated with uniform time discretization and Monte Carlo integration.
    Rewrite this loss in terms of the conditional vector field and the corresponding solutions to the continuity equation, and explain why it is equivalent to the flow matching loss above.
    (3 points)
    \item Choose $\alpha_t, \beta_t,$ and an optimal coupling between $X_0$ and $X_1$ as the distribution of $Z.$ Apply conditional flow matching for the specified target distribution.
    Attach a plot of the vector field $v_\theta$ at t=0, 0.5 and 1 and interpret your results. You are allowed to use rather than write your own implementation.
    (5 points)
    \item Is the vector field you obtain close to the solution of the Benamou-Brenier formulation? Why or why not? (3 points)
\end{enumerate}
\end{document}
