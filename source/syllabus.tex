\documentclass[12pt]{article}
\usepackage[tagged, highstructure]{accessibility}
\usepackage[english]{babel}
\usepackage[utf8x]{inputenc}
\usepackage[T1]{fontenc}
\usepackage[margin=1in]{geometry}
\usepackage{scribe}
\usepackage{listings}
\usepackage{natbib,verbatim}
\usepackage{hyperref}
\hypersetup{
    colorlinks=true,
    linkcolor=blue,
    filecolor=magenta,      
    urlcolor=magenta,
    pdftitle={Course Syllabus},
    pdfauthor={Nisha Chandramoorthy},
    pdflang={en-US}
}

%\Scribe{Your Name}
\title{Syllabus-37780}
\Lecturer{Nisha Chandramoorthy (nishac@uchicago.edu)}
\LectureNumber{\begin{large} Mathematics of Generative Models \end{large}}
\LectureDate{}
\LectureTitle{\begin{large}CAAM/STAT 37780\end{large}}

\lstset{style=mystyle}

\begin{document}
\MakeScribeTop

Generative models are machine learning algorithms that learn to approximately sample from a target probability distribution given a finite number of points sampled according to the target.
We use generative models or at least interact with their outputs every day. Generative models are entering every sphere of science and engineering quickly, but we do not 
completely understand what samples they produce, and a rigorous understanding is probably not possible. In this class, we instead try to learn some well-established facts and concepts that underlie the main types of generative modeling used today. 
You can see the tentative lecture schedule for the list of topics. We will assume a background in probability and some experience with machine learning.
\href{https://github.com/ni-sha-c/Mathematics-of-GMs}{Here} is the Github repo for this class.

\section{General information}
\begin{itemize}
	\item 2 80-minute lectures per week, 3 problem sets, 1 final project, 2 20-minute in-class quizzes.
	\item Class time and location: Tuesdays and Thursdays, 9:30 am -- 10:50 am, Eckhart 133
	\item Class dates: Jan 6, 2026 -- Mar 4, 2026.
	\item TA: Ziwei Su
	\item TA email: suziwei@uchicago.edu
	\item Instructor office hours: 3 pm -- 4 pm on Tuesdays
	\item Instructor email: nishac@uchicago.edu
\end{itemize}



\section{Grading information and late policy}

This is a discussion-oriented advanced topics course. The grade will be determined by a final project (40\%), 3 homeworks (30\%) and the 2 in-class quizzes (30\%). The final project will be a research project on a topic of your choice, related to the course material. I will assist you in the selection of a project and designing its scope, if needed. \\

\textbf{Quizzes}: These are quick in-class tests, typically with multiple choice or short-answer questions. 

\textbf{Final project}: The final project has to be done individually, and the deliverables include a proposal, code, and accompanying report. A final project rubric and a set of guidelines will be posted on canvas before the proposal due date. All written material should be typed up and submitted on Gradescope.\\

\textbf{Homeworks}: there will be 3 homework assignments (due dates on Canvas, spread out evenly through the quarter before the final project) that will be theoretical and often require numerical solutions. You are welcome to discuss with other students and use online resources, including AI assistants such as ChatGPT and Github CoPilot, to solve the questions. After that, however, all the submitted work should be your own. Please submit LaTeX-ed homework solutions (handwritten solutions are often illegible and will not be graded) on Gradescope as a pdf. \\


\textbf{Late policy}: you can submit late within 24 hours of the deadline without penalty, and after that, there is a late penalty of 25\% per day of delay.  



\section{List of topics}
\label{sec:topics}
The syllabus section on Canvas will have a lecture-wise breakdown of topics and will be updated through the quarter. 

\section{Accommodations for Students with Disabilities} 

If you are a student with learning needs that require special accommodation, contact the \href{https://disabilities.uchicago.edu}{Student Disability Services} (SDS). 
Please meet with me to discuss your access needs in this class after you have completed the SDS procedures for requesting accommodations.
SDS contact information: \\
Phone: 773-702-6000 \\
Email: disabilities@uchicago.edu. \\
Website: https://disabilities.uchicago.edu 

\section{Honor code}

We should all act with academic integrity, which means that 
we cite resources that we use, give due credit to other students or online/AI resources we have consulted, and submit only our own work.
Please see the \href{https://studentmanual.uchicago.edu/academic-policies/academic-honesty-plagiarism/}{University's policy} on academic honesty and 
refer to the \href{https://studentmanual.uchicago.edu/academic-policies/academic-honesty-plagiarism/}{student manual}.
%\bibliographystyle{abbrv}           % if you need a bibliography
%\bibliography{mybib}                % assuming yours is named mybib.bib


%%%%%%%%%%% end of doc
\end{document}
